% Resume for Andrew Fugier
% Orginal template was located at
% http://www.tedpavlic.com/post_resume_cv_latex_example.php
% I've left almost all of the orginal comments and whatnot so that if
% others like the general format, they may use this as their own template.

%%%%%%%%%%%%%%%%%%%%%%%%%%%% Document Setup %%%%%%%%%%%%%%%%%%%%%%%%%%%%
% Don't like 10pt? Try 11pt or 12pt
\documentclass[10pt]{article}
\RequirePackage[T1]{fontenc}

% LaTeX will typeset using Computer Modern Roman, which a lot of
% non-mathematicians and non-engineers won't like. Also, a few PDF
% viewers may not render CMR very well. Instead, Times New Roman can
% be used. That's what this package does.
\usepackage{times}

% The automated optical recognition software used to digitize resume
% information works best with fonts that do not have serifs. This
% command uses a sans serif font throughout. Uncomment both lines (or at
% least the second) to restore a Roman font (i.e., a font with serifs).
% (NOTE: This requires the times package above)
%\renewcommand{\familydefault}{\sfdefault}

% This is a helpful package that puts math inside length specifications
\usepackage{calc}

% This package helps LaTeX auto-hyphenate hyphenated words if you use
% special hyphens. For example, bio\-/mimicry will properly hyphenate
% ``mimicry'' if necessary.
\usepackage[shortcuts]{extdash}

% Layout: Puts the section titles on left side of page
\reversemarginpar

% PAPER SIZE, PAGE NUMBER, AND DOCUMENT LAYOUT NOTES:
% The next \usepackage line changes the layout for CV style section
% headings as marginal notes. It also sets up the paper size as either
% letter or A4. By default, letter was used. If A4 paper is desired,
% comment out the letterpaper lines and uncomment the a4paper lines.
%
% As you can see, the margin widths and section title widths can be
% easily adjusted.
%
% ALSO: Notice that the includefoot option can be commented OUT in order
% to put the PAGE NUMBER *IN* the bottom margin. This will make the
% effective text area larger.
%
% IF YOU WISH TO REMOVE THE ``of LASTPAGE'' next to each page number,
% see the note about the +LP and -LP lines below. Comment out the +LP
% and uncomment the -LP.
%
% IF YOU WISH TO REMOVE PAGE NUMBERS, be sure that the includefoot line
% is uncommented and ALSO uncomment the \pagestyle{empty} a few lines
% below.

%% Use these lines for letter-sized paper
\usepackage[paper=letterpaper,
            %includefoot, % Uncomment to put page number above margin
            marginparwidth=1.2in,     % Length of section titles
            marginparsep=.05in,       % Space between titles and text
            margin=1in,               % 1 inch margins
            includemp]{geometry}

%% Use these lines for A4-sized paper
%\usepackage[paper=a4paper,
%            %includefoot, % Uncomment to put page number above margin
%            marginparwidth=30.5mm,    % Length of section titles
%            marginparsep=1.5mm,       % Space between titles and text
%            margin=25mm,              % 25mm margins
%            includemp]{geometry}

%% More layout: Get rid of indenting throughout entire document
\setlength{\parindent}{0in}

% Provides special list environments and macros to create new ones
\usepackage[shortlabels]{enumitem}

%% Reference the last page in the page number
%
% NOTE: comment the +LP line and uncomment the -LP line to have page
%       numbers without the ``of ##'' last page reference)
%
% NOTE: uncomment the \pagestyle{empty} line to get rid of all page
%       numbers (make sure includefoot is commented out above)
%
\usepackage{fancyhdr,lastpage}
\pagestyle{fancy}
%\pagestyle{empty}      % Uncomment this to get rid of page numbers
\fancyhf{}\renewcommand{\headrulewidth}{0pt}
\fancyfootoffset{\marginparsep+\marginparwidth}
\newlength{\footpageshift}
\setlength{\footpageshift}
          {0.5\textwidth+0.5\marginparsep+0.5\marginparwidth-2in}
\lfoot{\hspace{\footpageshift}%
       \parbox{4in}{\, \hfill %
                    \arabic{page} of \protect\pageref*{LastPage} % +LP
%                    \arabic{page}                               % -LP
                    \hfill \,}}

% Finally, give us PDF bookmarks
\usepackage{color,hyperref}
\definecolor{darkblue}{rgb}{0.0,0.0,0.3}
\hypersetup{colorlinks,breaklinks,
            linkcolor=darkblue,urlcolor=darkblue,
            anchorcolor=darkblue,citecolor=darkblue}
%%%%%%%%%%%%%%%%%%%%%%%% End Document Setup %%%%%%%%%%%%%%%%%%%%%%%%%%%%

%%%%%%%%%%%%%%%%%%%%%%%%%%% Helper Commands %%%%%%%%%%%%%%%%%%%%%%%%%%%%
%%% HEADING AT TOP OF CURRICULUM VITAE
% The title (name) with a horizontal rule under it
% (optional argument typesets an object right-justified across from name
%  as well)
%
% Usage: \makeheading{name}
%        OR
%        \makeheading[right_object]{name}
%
% Place at top of document. It should be the first thing.
% If ``right_object'' is provided in the square-braced optional
% argument, it will be right justified on the same line as ``name'' at
% the top of the CV. For example:
%
%       \makeheading[\emph{Curriculum vitae}]{Your Name}
%
% will put an emphasized ``Curriculum vitae'' at the top of the document
% as a title. Likewise, a picture could be included:
%
%   \makeheading[{\includegraphics[height=1.5in]{my_picture}}]{Your Name}
%
% the picture will be flush right across from the name. For this example
% to work, make sure the extra set of curly braces is included. Also
% makes ure that \usepackage{graphicx} is somewhere in the preamble.
\newcommand{\makeheading}[2][]%
        {\hspace*{-\marginparsep minus \marginparwidth}%
         \begin{minipage}[t]{\textwidth+\marginparwidth+\marginparsep}%
             {\large \bfseries #2 \hfill #1}\\[-0.15\baselineskip]%
                 \rule{\columnwidth}{1pt}%
         \end{minipage}}

%%% SECTION HEADINGS
% The section headings. Flush left in small caps down pseudo-margin.
%
% Usage: \section{section name}
\renewcommand{\section}[1]{\pagebreak[3]%
    \vspace{1.3\baselineskip}%
    \phantomsection\addcontentsline{toc}{section}{#1}%
    \noindent\llap{\scshape\smash{\parbox[t]{\marginparwidth}{\hyphenpenalty=10000\raggedright #1}}}%
    \vspace{-\baselineskip}\par}

%%% LISTS
% This macro alters a list by removing some of the space that follows the list
% (is used by lists below)
\newcommand*\fixendlist[1]{%
    \expandafter\let\csname preFixEndListend#1\expandafter\endcsname\csname end#1\endcsname
    \expandafter\def\csname end#1\endcsname{\csname preFixEndListend#1\endcsname\vspace{-0.6\baselineskip}}}

% These macros help ensure that items in outer-type lists do not get
% separated from the next line by a page break
% (they are used by lists below)
\let\originalItem\item
\newcommand*\fixouterlist[1]{%
    \expandafter\let\csname preFixOuterList#1\expandafter\endcsname\csname #1\endcsname
    \expandafter\def\csname #1\endcsname{\let\oldItem\item\def\item{\pagebreak[2]\oldItem}\csname preFixOuterList#1\endcsname}
    \expandafter\let\csname preFixOuterListend#1\expandafter\endcsname\csname end#1\endcsname
    \expandafter\def\csname end#1\endcsname{\let\item\oldItem\csname preFixOuterListend#1\endcsname}}
\newcommand*\fixinnerlist[1]{%
    \expandafter\let\csname preFixInnerList#1\expandafter\endcsname\csname #1\endcsname
    \expandafter\def\csname #1\endcsname{\let\oldItem\item\let\item\originalItem\csname preFixInnerList#1\endcsname}
    \expandafter\let\csname preFixInnerListend#1\expandafter\endcsname\csname end#1\endcsname
    \expandafter\def\csname end#1\endcsname{\csname preFixInnerListend#1\endcsname\let\item\oldItem}}

% An itemize-style list with lots of space between items
%
% Usage:
%   \begin{outerlist}
%       \item ...    % (or \item[] for no bullet)
%   \end{outerlist}
\newlist{outerlist}{itemize}{3}
    \setlist[outerlist]{label=\enskip\textbullet,leftmargin=*}
    \fixendlist{outerlist}
    \fixouterlist{outerlist}

% An environment IDENTICAL to outerlist that has better pre-list spacing
% when used as the first thing in a \section
%
% Usage:
%   \begin{lonelist}
%       \item ...    % (or \item[] for no bullet)
%   \end{lonelist}
\newlist{lonelist}{itemize}{3}
    \setlist[lonelist]{label=\enskip\textbullet,leftmargin=*,partopsep=0pt,topsep=0pt}
    \fixendlist{lonelist}
    \fixouterlist{lonelist}

% An itemize-style list with little space between items
%
% Usage:
%   \begin{innerlist}
%       \item ...    % (or \item[] for no bullet)
%   \end{innerlist}
\newlist{innerlist}{itemize}{3}
    \setlist[innerlist]{label=\enskip\textbullet,leftmargin=*,parsep=0pt,itemsep=0pt,topsep=0pt,partopsep=0pt}
    \fixinnerlist{innerlist}

% An environment IDENTICAL to innerlist that has better pre-list spacing
% when used as the first thing in a \section
%
% Usage:
%   \begin{loneinnerlist}
%       \item ...    % (or \item[] for no bullet)
%   \end{loneinnerlist}
\newlist{loneinnerlist}{itemize}{3}
    \setlist[loneinnerlist]{label=\enskip\textbullet,leftmargin=*,parsep=0pt,itemsep=0pt,topsep=0pt,partopsep=0pt}
    \fixendlist{loneinnerlist}
    \fixinnerlist{loneinnerlist}

%%% EXTRA SPACE
% To add some paragraph space between lines.
% This also tells LaTeX to preferably break a page on one of these gaps
% if there is a needed pagebreak nearby.
\newcommand{\blankline}{\quad\pagebreak[3]}
\newcommand{\halfblankline}{\quad\vspace{-0.5\baselineskip}\pagebreak[3]}

%%% FORMATTING MACROS
% Provides a linked \doi{#1} that links doi:#1 to http://dx.doi.org/#1
\usepackage{doi}
% To change the text before the DOI, adjust this command
%\renewcommand\doitext{doi:}

% Provides a linked \url{#1} that doesn't require escape characters
\usepackage{url}

% You can adjust the style \url{} uses here:
% (options are: same, rm, sf, tt; defaults to tt)
\urlstyle{same}

% For \email{ADDRESS}, links ADDRESS to the url mailto:ADDRESS
% (uncomment to typeset the e\-/mail address in typewriter font;
%  otherwise, will be typeset in the \urlstyle above)
%\DeclareUrlCommand\emaillink{\urlstyle{tt}}
\providecommand*\emaillink[1]{\nolinkurl{#1}}
\providecommand*\email[1]{\href{mailto:#1}{\emaillink{#1}}}

\providecommand\BibTeX{{B\kern-.05em{\sc i\kern-.025em b}\kern-.08em \TeX}}
\providecommand\Matlab{\textsc{Matlab}}

% Custom hyphenation rules for words that LaTeX has trouble with
\hyphenation{re-us-a-ble pro-vid-er Media-Wiki}

%%%%%%%%%%%%%%%%%%%%%%%% End Helper Commands %%%%%%%%%%%%%%%%%%%%%%%%%%%

%%%%%%%%%%%%%%%%%%%%%%%%% Begin CV Document %%%%%%%%%%%%%%%%%%%%%%%%%%%%

\begin{document}
\makeheading{Andrew Fugier}

\section{Contact Information}

% NOTE: Mind where the & separators and \\ breaks are in the following
%       table. Table is one row made up of three parboxes. The left
%       parbox has address info, the middle parbox has a vertical bar,
%       and the right parbox has phone and electronic contact
%       information.
%
% MACROS: \rcollength is the width of the right column of the table
%             (adjust it to your liking; default is 1.85in).
%         \spacewidth is width of area between left and right boxes.

\newlength{\rcollength}\setlength{\rcollength}{2.0in}%
\newlength{\spacewidth}\setlength{\spacewidth}{20pt}

\begin{tabular}[t]{@{}p{\textwidth-\rcollength-\spacewidth}@{}p{\spacewidth}@{}p{\rcollength}}

% Address box
\parbox{\textwidth-\rcollength-\spacewidth}{
Andrew Fugier\\
1605 Terrace Ave\\
Snohomish WA, 98290}

&
\parbox[m][5\baselineskip]{\spacewidth}{} &

% Non-snail-mail contact information
\parbox{\rcollength}{
\textit{Mobile:} +1 (425) 239-8506 \\
\textit{E-mail:} \email{fugier@gmail.com}}

\end{tabular}

\section{Skills}

\begin{innerlist}
  \item[] \textbf{Programming Languages}
  \begin{innerlist}
    \item Professional: C, C\#, other .NET platforms (ASP, VB, etc.)
    \item Academic: Java, Python, C++, Perl, Ruby, PHP, Scheme
    \item Ability to quickly learn new languages, techniques
  \end{innerlist}

  \item[] \textbf{Source Control Experience}
  \begin{innerlist}
    \item Git, SVN, Microsoft Team Foundation Server
    \item CVS: Academic usage
  \end{innerlist}
  
  \item[] \textbf{University Elective Coursework}
  \begin{innerlist}
    \item Parallel Computation: Use of compute cluster / MPI / Infniband / MapReduce
    \item Mobile Device Programming: Writing native apps for the Android platform
    \item Advanced Web Programming in Java: More in-depth look at Java
    \item Internet Studies Internship: Wrote a web API using Java / MySQL / JSON
  \end{innerlist}

  \item[] \textbf{Microsoft Certified Professional, Microsoft Certified Technology Specialist}
  \begin{innerlist}
    \item Certified Since 2005 - MCP\# 3380893
    \item Validation: Transcript ID 1135400 | Access Code 17ELN89P
  \end{innerlist}

  \item[] \textbf{IT Experience: 15+ years Windows experience}
  \begin{innerlist}
    \item Desktop: Every major version from NT 4.0 through Windows 10
    \item Server: Every major version from NT 4.0 through Server 2016
    \item Hyper V virtualization services
    \item Exchange Server 2003 through recent (SBS and Standard Versions)
    \item SQL Server 2000 through recent
	\item Active Directory \& Group Policy management experience
  \end{innerlist}

  \item[] \textbf{IT Experience: 9+ years Linux / FreeBSD experience}
  \begin{innerlist}
    \item ZFS, NFS, SMB, CIFS, DHCP, PXE, TFTP services
    \item Use of MySQL, PHP, Apache in production environments
    \item Support of 200+ Linux workstations, at university - 2010 though 2014
    \item FreeBSD based systems, such as Isilon, FreeNAS, pfSense, etc.
  \end{innerlist}

  \item[] \textbf{Other interests I have and interesting things I've worked with}
  \begin{innerlist}
    \item Small ARM based systems, such as the Raspberry Pi, BeagleBone
    \item Microcontroller systems, mainly Atmel 328 series (Arduino based)
    \item ZigBee / XBee wireless systems / wireless serial communications
    \item Projects involving electronics, sensors, and data computation
    \item Various 3D printing and CNC wood working machines
  \end{innerlist}

  \item[] \textbf{Working to provide solutions in technology for the past 20 years}
\end{innerlist}


\section{Education}

\href{http://www.wwu.edu/}{\textbf{Western Washington University}}, Bellingham, WA
\begin{innerlist}
  \item[] B.S., \href{https://cse.wwu.edu/computer-science}{Computer Science}, August 2014
  \begin{innerlist}
    \item Elective coursework around Unix, parallel computation, and mobile device development
    \item Certificate from the \href{http://yorktown.cbe.wwu.edu/isc/}
          {Internet Studies Center} (Web Programmer Track)
  \end{innerlist}
\end{innerlist}

\medskip

\href{http://northseattle.edu/}{\textbf{North Seattle Community College}}, Seattle, WA
\begin{innerlist}
  \item[] A.A., General Studies, August 2010
  \begin{innerlist}
    \item Achieved Dean's List status multiple quarters
  \end{innerlist}
\end{innerlist}

\medskip
\medskip
\medskip
\medskip

\section{Experience}

\href{https://www.emc.com/en-us/storage/isilon/}{\textbf{Isilon Systems}}\textbf{ part of \href{https://www.emc.com/en-us/}{Dell | EMC}}, Seattle, WA
\begin{outerlist}
  \item[] \textit{Principal Software Quality Engineer, Senior Software Quality Engineer}
  \hfill \textbf{2014 - Present}
  \begin{innerlist}
  \item Determine cause of software failures in the field, specifically in these areas
  \begin{innerlist}
    \item AIMA (Authentication, Identity Management, and Access control)
    \item File Protocols - specifically SMB, CIFS, NFS, NFSv4, FTP
    \item Networking components including Ethernet and Infiniband stacks
  \end{innerlist}
  \item Analysis of program cores using GDB and similar tools
  \item Determining protocol adherence based on network packet captures
  \end{innerlist}
\end{outerlist}

\halfblankline

\href{http://www.wwu.edu/}{\textbf{Western Washington University}}, Bellingham, WA
\begin{outerlist}
  \item[] \textit{Systems Administrator, Computer Science Department}
  \hfill \textbf{2012 - 2014}
  \begin{innerlist}
    \item Assisted the Senior System Analyst/Programmer in tasks and projects
    \item Experience with FreeBSD, especially with ZFS, NFS, and Samba
    \item Supported a large user base (2000+) across Windows and Unix / Linux systems
    \item Installed \& assisted the configuration of a high performance computing cluster
    \begin{innerlist}
      \item Experience with Rocks cluster management software (Linux Based)
      \item Experience with IPMI Management Interfaces
      \item Experience with Infiniband networking
    \end{innerlist}
    \item Cisco networking equipment exposure
    \begin{innerlist}
      \item Designed, implemented, and configured 20Gb/s backbone for the 
            department, \\with over 400 individual gigabit service ports.
      \item Experience in cross integrating technology between Cisco and 3rd party
            equipment, in particular, Netgear and HP. (LLDP, Link Aggregation, etc.)
    \end{innerlist}
    \item Installed and configured large storage systems (~50TB usable)
    \begin{innerlist}
      \item 24+ Disk systems, multi-level ZFS, 200+ simultaneous connections
      \item Backed by FreeBSD and ZFS, available through NFS and Samba
    \end{innerlist}
  \end{innerlist}
\end{outerlist}

\halfblankline

\href{http://www.valuelogic.com/}{\textbf{Value Logic}}, Vista, CA
\begin{outerlist}
  \item[] \textit{Software Engineer}
  \hfill (2010 - Present: Part-time contract)
  \hfill \textbf{2008 - 2014}
  \begin{innerlist}
  \item Developed and maintained internal ticketing software (Service Ticket Plus)
  \begin{innerlist}
    \item Software developed in C\#, with MySQL, MSSQL, SQLite databases used
    \item WinForms for user interfaces, classes for business logic
  \end{innerlist}
  \item Extended exposure to Team Foundation Server
  \item Exposure to thrid-party library and driver usage, vendor specific hardware
  \end{innerlist}
\end{outerlist}

\halfblankline

\href{http://www.valuelogic.com/}{\textbf{Value Logic}}, Vista, CA
\begin{outerlist}
  \item[] \textit{IT Administrator}
  \hfill \textbf{2003 - 2009}
  \\Job responsibilities have changed greatly over time; basic list (In-Office and On-Site):
  \begin{innerlist}
    \item Day to day maintenance of approximately twenty small business networks
    \item Manage a small team of system engineers, keeping idle time to a minimum
    \item Responsible for server configurations, builds, RAID mgt., O/S loads
    \item Responsible for deployment of Active Directory, Exchange, DNS, DHCP, etc.
    \item Responsible for configuration of SonicWALL brand firewalls / routers
    \item Responsible for WAN link site-to-site setup and optimizations in organizations
    \item First responder for system wide downtime / failures - 24 hour on-call
    \item Implement special customer requests, for example, POS systems, Wi-Fi barcode
    \\    scanners, distributed wireless systems, custom programming, reports, etc.
  \end{innerlist}
\end{outerlist}

\halfblankline

\end{document}

%%%%%%%%%%%%%%%%%%%%%%%%%% End CV Document %%%%%%%%%%%%%%%%%%%%%%%%%%%%%
% The following is copyright and licensing information for
% redistribution of this LaTeX source code; it also includes a liability
% statement. If this source code is not being redistributed to others,
% it may be omitted. It has no effect on the function of the above code.
% Copyright (c) 2007, 2008, 2009, 2010, 2011 by Theodore P. Pavlic
%
% Unless otherwise expressly stated, this work is licensed under the
% Creative Commons Attribution-Noncommercial 3.0 United States License. To
% view a copy of this license, visit
% http://creativecommons.org/licenses/by-nc/3.0/us/ or send a letter to
% Creative Commons, 171 Second Street, Suite 300, San Francisco,
% California, 94105, USA.
%
% THE SOFTWARE IS PROVIDED "AS IS", WITHOUT WARRANTY OF ANY KIND, EXPRESS
% OR IMPLIED, INCLUDING BUT NOT LIMITED TO THE WARRANTIES OF
% MERCHANTABILITY, FITNESS FOR A PARTICULAR PURPOSE AND NONINFRINGEMENT.
% IN NO EVENT SHALL THE AUTHORS OR COPYRIGHT HOLDERS BE LIABLE FOR ANY
% CLAIM, DAMAGES OR OTHER LIABILITY, WHETHER IN AN ACTION OF CONTRACT,
% TORT OR OTHERWISE, ARISING FROM, OUT OF OR IN CONNECTION WITH THE
% SOFTWARE OR THE USE OR OTHER DEALINGS IN THE SOFTWARE.
